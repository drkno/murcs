\section{Logging Effort}

If you've done some work its a good idea to log it. In order to do this with murcs, simply navigate to that task you've been working on and press the log button (circled below).

\begin{figure}[H]
\centering
\includegraphics[width=\textwidth]{images/screenshots/logging1.PNG}
\caption{The Log Button}
\label{fig:new_project}
\end{figure}

You will be greeted with the logging effort popover. To log some effort simply enter it into the top form and press the "+" button (circled below). 

\begin{figure}[H]
\centering
\includegraphics[width=\textwidth]{images/screenshots/logging3.png}
\caption{The Add Button}
\label{fig:new_project}
\end{figure}

If you wish to enter an effort entry where more than one person has worked on the task (peer programming) then you simply click the add button and select the people who worked on the task together.

\begin{figure}[H]
\centering
\includegraphics[width=\textwidth]{images/screenshots/peerprogramming1.png}
\caption{Peer Programming}
\label{fig:new_project}
\end{figure}

Your effort will be added to the task and show up on burndowns. Effort requires a date, description, person logging effort and an amount of time (the effort). Fields with a problem will be highlighted in red.

\begin{figure}[H]
\centering
\includegraphics[width=\textwidth]{images/screenshots/logging2.png}
\caption{Form Errors}
\label{fig:new_project}
\end{figure}

All logged effort for the task shows up in the "Spent Effort" section of this form. 

\begin{figure}[H]
\centering
\includegraphics[width=\textwidth]{images/screenshots/logging4.png}
\caption{Spent Effort}
\label{fig:new_project}
\end{figure}

Effort can be removed using the delete button (circled below).

\begin{figure}[H]
\centering
\includegraphics[width=\textwidth]{images/screenshots/logging5.png}
\caption{The Remove Button}
\label{fig:new_project}
\end{figure}