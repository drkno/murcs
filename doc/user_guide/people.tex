\section{People Maintenance}

Managing People
\newline\newline
People can be easily added to the list of unassigned people and then assigned to different teams as necessary.
\newline
To start with creating new people is very simple. There are two ways of doing it, the first is to make sure you have the People selected in the display list as shown below. Then click the + button in the bottom as labelled in the figure below.

\begin{figure}[H]
\centering
\includegraphics[width=\textwidth]{images/screenshots/people1.PNG}
\caption{Adding a Person Method 1}
\label{fig:new_project}
\end{figure}

The other method is to select File/New/Person from the File menu at the top of the application as shown in the fig below.

\begin{figure}[H]
\centering
\includegraphics[width=\textwidth]{images/screenshots/people4.PNG}
\caption{Adding a Person Method 2}
\label{fig:new_project}
\end{figure}

Once you have clicked the add button or selected file/new/person a dialog will appear that asks for information about the new person you are creating. The minimum requirements for this are the Name and UserID, all of the other fields are optional. These fields include, a full name and a method for addding skills to the person.

\begin{figure}[H]
\centering
\includegraphics[width=\textwidth]{images/screenshots/people2.PNG}
\caption{Person Creation Dialog}
\label{fig:new_project}
\end{figure}

To add a skill simply select it from the drop down list (this is populated based on the skills you have added already) and it will appear in the list of skills. Then to remove a skill simply click x next to the skill.

\begin{figure}[H]
\centering
\includegraphics[width=\textwidth]{images/screenshots/people3.PNG}
\caption{Person adding/removing Skills}
\label{fig:new_project}
\end{figure}

In order to delete a person simply select it from the side list and click the X button next to the add button in the list display.

Once you have a person created you can add them to Teams which will be covered in the Teams section of this user guide.