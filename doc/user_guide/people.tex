\section{People Maintenance}

Managing People

\begin{figure}[H]
	\centering
	\includegraphics[width=\textwidth]{images/screenshots/people1.PNG}
	\caption{Adding a Person Method 1}
	\label{fig:new_project}
\end{figure}

People can be easily added to Murcs. 
\newline
Creating a new person is very simple. There are two ways of doing it, the first is to make sure you have the Person selected in the display list as shown below. Then click the new button on the toolbar.

\begin{figure}[H]
	\centering
	\includegraphics[width=\textwidth]{images/screenshots/people4.PNG}
	\caption{Adding a Person Method 2}
	\label{fig:new_project}
\end{figure}

The other method is to select File-\textgreater New-\textgreater Person from the File menu at the top of the application.

\begin{figure}[H]
	\centering
	\includegraphics[width=\textwidth]{images/screenshots/people2.PNG}
	\caption{Person Creation Dialog}
	\label{fig:new_project}
\end{figure}

Once you have clicked the add button or selected File-\textgreater New-\textgreater Person a dialog will appear that asks for information about the new person you are creating. The minimum requirements for this are the Name and UserID, all of the other fields are optional. These fields include, a full name and a method for adding skills to the person.

\begin{figure}[H]
	\centering
	\includegraphics[width=\textwidth]{images/screenshots/people3.PNG}
	\caption{Person adding/removing Skills}
	\label{fig:new_project}
\end{figure}

To add a skill simply select it from the drop down list (this is populated based on the skills you have added already) and it will appear in the list of skills. Then to remove a skill simply click Faded X delete button next to the skill. Note: If you add the PO skill to a person and then assign them to be PO of a Team and then remove the PO skill from them then they will be removed from the position of PO on that team.

In order to delete a person simply select it from the side list and click the delete button from the toolbar. When a person is deleted it will be removed from any place it is referenced in the application.

Once you have a person created you can add them to Teams which is covered in the Teams section of this user guide.