\section{Element Deletion}

Elements can easily be deleted using the small 'X' button found at the bottom left of the application, underneath the Item List. You can also press ctrl + delete when you have selected the item you want to delete. Following is a simple walk through for deleting an element.

First, choose the type of the element you wish to delete. This can be either Projects, Teams, People or Skills and select it in the Display Choice Picker (circled below).

\begin{figure}[H]
\centering
\includegraphics[width=\textwidth]{images/screenshots/deletion1.PNG}
\caption{The Display Choice Picker}
\label{fig:new_project}
\end{figure}

In our case, we've chosen to delete a person named Pura Hayden. The next step is to select this person and press the delete button (circled below).

\begin{figure}[H]
\centering
\includegraphics[width=\textwidth]{images/screenshots/deletion2.PNG}
\caption{The Delete Button}
\label{fig:new_project}
\end{figure}

The final step is to confirmation. You will be a presented with a message asking you if you are really sure you want to go through with a deletion along with a list of places that the element you are deleting is used. Press the 'Yes' button (circled) to go through with the deletion. If you've changed your mind you can click the 'No' button. 

\begin{figure}[H]
\centering
\includegraphics[width=\textwidth]{images/screenshots/deletion3.PNG}
\caption{The Confirmation Dialog. It seems Pura is part of a team named 'Bar'}
\label{fig:new_project}
\end{figure}

Deleting Pura also removes her from any teams she might have been part of, so be careful!

If you've deleted something by mistake, don't worry! Deletions can be undone. Phew.